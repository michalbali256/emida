\chapter*{Introduction}
\addcontentsline{toc}{chapter}{Introduction}
Electron backscatter diffraction (EBSD) is a scientific tool used to examine crystalline materials. It is based on electron diffraction: when a beam of electrons is emitted towards the object of interest, some of the electrons backscatter. They are then captured by a camera, resulting in a grayscale raster image of the \emph{backscatter pattern}.

The crystalline structure deforms under stress, which results in deformation of the backscatter pattern. We can analyze the deformation of the pattern to determine several characteristics of the crystalline material such as crystal orientation, phase or elastic strain.

In 2006, Wilkinson, Meaden and Dingley described the first technique of EBSD analysis. It is based on cross correlating several sub--regions  of the deformed and reference (undeformed) images. The cross correlation is processed to obtain shifts between the regions of interest with subpixel accuracy by interpolating between the points of best correlation match. From the shifts, it is possible to determine elastic strain and lattice rotation of the crystalline material.

When examining an object using EBSD, the procedure has to be repeated thousands of times. Usually, the images are taken in a raster that covers the area of interest which produces thousands of images. For each image, the analysis involves cross correlating tens to hundreds of subregions (100x100 pixels) which computationally expensive. At the same time, processing of individual subregions is independent that makes it possible to leverage data parallelism and thus is appropriate for implementation on modern GPUs. 

In this thesis we analyze the technique used to process data from EBSD from performance point of view, implement the most computationally expensive parts using the CUDA technology and benchmark the implementation. \todo{to be finished...}
