\chapter*{Introduction}
\addcontentsline{toc}{chapter}{Introduction}

Physics is constantly developing new methods how to describe and measure the reality around us. Nowadays, it is easily possible to collect vast amount of data, however it is useless unless we are able to analyze it efficiently. This thesis aims to help analysis of data used to study the microstructure of materials.

Electron backscatter diffraction (EBSD) is a scientific tool used to examine crystalline materials. It is based on an electron microscope shooting a beam of electrons towards the studied specimen. They are then reflected off the surface and captured by a camera, resulting in a grayscale image of the \emph{backscatter pattern}. Physicists then study the deformation of those patterns to determine several characteristics of the crystalline material such as crystal orientation, phase or elastic strain.

In 2006, Wilkinson, Meaden and Dingley first described the technique of high resolution EBSD analysis \cite{wilkinson2006high}. It is based on cross--correlating several subregions of a deformed and reference (undeformed) images. The cross-correlation is then processed to obtain shifts between the regions of interest with subpixel accuracy by interpolating between the points of best correlation match. From the shifts, it is possible to determine elastic strain and lattice rotation of the crystalline material.

When examining an object using EBSD, the procedure has to be repeated thousands or even millions of times. Usually, the images are taken in a raster that covers the area of interest which produces thousands of images. For each image, the analysis involves cross correlating tens of subregions ($100\times100$ pixels), which is computationally expensive. At the same time, processing of individual subregions is independent that makes it possible to leverage data parallelism and thus is appropriate for implementation on modern GPUs.

Faster processing of patterns enables the physicists to process bigger datasets that cover larger areas with higher resolution. Current commercially available software is insufficient for larger EBSD maps, which can contain up to million of patterns and the analysis takes tens of hours. A fast GPU implementation has the potential to drastically improve the situation.

Modern EBSD cameras are also able to take hundreds of images per second depending on image quality. That means it is possible to create large datasets in a reasonable amount of time and the analysis is the bottleneck of the process. Reference python implementation, that has been provided by physicists from Department of Physics of Material at Charles University in Prague, can process from 5 to 50 patterns per second, depending on the input parameters.

In this thesis we describe the technique used to process the EBSD data in detail, analyze it from performance point of view, implement the most computationally expensive parts using the CUDA technology and benchmark the implementation.

The thesis is organized as follows. \Cref{chap1} offers a detailed explanation of the algorithm of interest. We first define the cross--correlation in \cref{cross-corr-def}, show how it can be computed using the discrete Fourier transform. In \cref{\subpixel-peak}, we explain how to process the cross--correlation result to obtain the desired vectors of shift between the subregions. In \cref{chap2}, we analyze the algorithm from performance point of view and outline its GPU implementation. First, we explain three GPU kernels that execute required steps (cross--correlation, sum and arg max). Then, in \cref{preprocessing} and \cref{cross-processing}, we describe how they are put together to implement the algorithm. Finally, we outline how we can parallelize loading of the patterns from disk and their processing on the GPU. In \cref{chap3}, we measure the behavior of important parts of the implementation for different parameters and compare the python and GPU implementations.


